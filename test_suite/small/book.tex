% LaTeX source for ``Think Stats:
% Probability and Statistics for Programmers''
% Copyright 2011  Allen B. Downey.

% License: Creative Commons Attribution-NonCommercial 3.0 Unported License.
% http://creativecommons.org/licenses/by-nc/3.0/
%

%\documentclass[10pt,b5paper]{book}
\documentclass[12pt]{book}
\usepackage[width=5.5in,height=8.5in,
  hmarginratio=3:2,vmarginratio=1:1]{geometry}

% for some of these packages, you might have to install
% texlive-latex-extra (in Ubuntu)

\usepackage{pslatex}
\usepackage{url}
\usepackage{fancyhdr}
\usepackage{graphicx}
\usepackage{subfig}
\usepackage{amsmath}
\usepackage{amsthm}
%\usepackage{amssymb}
\usepackage{makeidx}
\usepackage{setspace}
\usepackage{hevea}                           
\usepackage{upquote}
\usepackage{db45}

\title{Think Stats}
\author{Allen B. Downey}
\newcommand{\thetitle}{Think Stats: Probability and Statistics for Programmers}
\newcommand{\theversion}{1.4.6}
\makeindex
\begin{document}
\frontmatter
\maketitle
\chapter*{Preface}
\label{preface}

\begin{exercise}

In most foot races, everyone starts at the same time.  If you are a
fast runner, you usually pass a lot of people at the beginning of the
race, but after a few miles everyone around you is going at the same
speed.
\index{relay race}
\index{bias!oversampling}
\index{oversampling}

When I ran a long-distance (209 miles) relay race for the first
time, I noticed an odd phenomenon: when I overtook another runner, I
was usually much faster, and when another runner overtook me, he was
usually much faster.

At first I thought that the distribution of speeds might be bimodal;
that is, there were many slow runners and many fast runners, but few
at my speed.

Then I realized that I was the victim of selection bias.  The race
was unusual in two ways: it used a staggered start, so teams started
at different times; also, many teams included runners at different
levels of ability.
\index{bias!selection}
\index{selection bias}

As a result, runners were spread out along the course with little
relationship between speed and location.  When I started running my
leg, the runners near me were (pretty much) a random sample of the
runners in the race.

So where does the bias come from?  During my time on the course, the
chance of overtaking a runner, or being overtaken, is proportional to
the difference in our speeds.  To see why, think about the extremes.
If another runner is going at the same speed as me, neither of us will
overtake the other.  If someone is going so fast that they cover the
entire course while I am running, they are certain to overtake me.

Write a function called {\tt BiasPmf} that takes a Pmf representing
the actual distribution of runners' speeds, and the speed of a running
observer, and returns a new Pmf representing the distribution of
runners' speeds as seen by the observer.

To test your function, get the distribution of speeds from a
normal road race (not a relay).  I wrote a program that reads the
results from the James Joyce Ramble 10K in Dedham MA and converts the
pace of each runner to MPH.  Download it from
\url{thinkstats.com/relay.py}.  Run it and look at the PMF of
speeds.
\index{{\tt relay.py}}
\index{{\tt relay\_soln.py}}

Now compute the distribution of speeds you would observe if you ran a
relay race at 7.5 MPH with this group of runners.  You can download a
solution from \url{thinkstats.com/relay_soln.py}

\end{exercise}


\printindex

\clearemptydoublepage
%\blankpage
%\blankpage
%\blankpage


\end{document}

%Chapter 10: time series

%serial correlation

%auto-correlation function

%generating random series with correlation

%violating the CLT and see if the sum of correlated normals is lognormal

%If so, maybe that explains adult weight distribution.




% Topics to consider in the future:

% Time series

% Correlation of adjacent elements

% Autocorrelation function

% Information theory: Paul Revere example

% Gelman's paradox
% \url{http://www.iq.harvard.edu/blog/sss/archives/2008/04/gelmans_paradox.shtml}

% regression to the mean experiment

% Poisson distribution

% example using csvDictReader


